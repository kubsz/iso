\documentclass{article}
\usepackage[utf8]{inputenc}
\usepackage{listings}
\usepackage{xcolor}
\usepackage{appendix}
\definecolor{backcolour}{rgb}{0.95,0.95,0.92}
\definecolor{keyword}{rgb}{100,0,100}

\usepackage[acronym]{glossaries}
\usepackage{graphicx}

\makeglossaries

\newglossaryentry{function}
{
        name=function,
        description={A function is a unit of code that is often defined by its role within a greater code structure}
}

\newglossaryentry{unix}
{
        name=unix time,
        description={Also known as seconds since epoch, unix time is a system to describe a point in time. It is the number of seconds that have elapsed since 00:00:00 Thursday, 1  January 1970.}
}

\newglossaryentry{mysql}
{
        name=mysql,
        description={mySQL is an open source relational database management system}
}

\newacronym{pk}{PK}{Primary Key}
\newacronym{fk}{FK}{Foreign Key}
\newacronym{sql}{SQL}{Structured Query Language}
\newacronym{php}{PHP}{Personal Home Page / Hypertext Preprocessor}
\newacronym{asc}{ASC}{Ascending}
\newacronym{desc}{DESC}{Descending}


\lstdefinestyle{mystyle}{
    basicstyle=\fontfamily{pcr}\selectfont\footnotesize,
    backgroundcolor=\color{backcolour},
    keywordstyle=\color{keyword},
    numbers=left,
    stringstyle=,
    showspaces=false,
    showstringspaces=false
}

\title{ISO Individual Report}
\author{
Kubilay Korkmaz\\
\texttt{me@kubs.uk}\\
\texttt{15114192@shrewsbury.ac.uk}
}
\date{March 2019}

\begin{document}

\maketitle

\clearpage

\tableofcontents

\clearpage

\section{Introduction}

\section{Useful Operational Information for a Teacher}
\subsection{Types of Information}In order for a teacher to operate at a functional level, there is specific information that is mandatory, and some information that just makes their lives easier on the job. For example - student names is absolutely mandatory due to the fact that a register must be completed at the start of every lesson. However, a list of student's parent contact information isn't necessary to have at all times, as this information can be easily queried from a higher member of staff, or school software. 

\section{Useful Tactical Information for a Faculty Manager}

\section{Useful Strategic Information for one of the Management Team}

\newpage

\printglossary[type=\acronymtype]
 
\printglossary

\lstlistoflistings

\end{document}
